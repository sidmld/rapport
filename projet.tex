\documentclass[french]{beamer}

\usepackage[utf8]{inputenc}
\usepackage[T1]{fontenc}
\usepackage{lmodern}
\usepackage{amsmath, amssymb}

\usepackage{babel}


%CHOIX DU THEME et/ou DE SA COULEUR
% => essayer différents thèmes (en décommantant une des trois lignes suivantes)
\usetheme{PaloAlto}
%\usetheme{Madrid}
%\usetheme{Copenhagen}

% => il est possible, pour un thème donné, de modifier seulement la couleur
\usecolortheme{crane}
%\usecolortheme{seahorse}

%\useoutertheme[left]{sidebar}


%Pour le TITLEPAGE
\title{Utilisation d'une approche de Machine Learning pour l'extraction des descripteurs d'un visage}
\subtitle{Projet semesteriel}
\author[]{BENHELLAL Narimane \\ MAOULOUD Sidi \\ ABDENOUR Idir}
\date{Mars 2017}
\institute[UT3 -- FSI]{Université du Havre -- Faculté des sciences et Techniques}


\begin{document}

\begin{frame}
	\titlepage
\end{frame}

\begin{frame}
	Un environnement \texttt{frame} pour chaque \emph{diapositive}.
	\visible<2>{Chaque diapo pouvant contenir plusieurs \emph{couches}.}
\end{frame}


\begin{frame}{On peut mettre un titre : Sommaire}
	\tableofcontents
\end{frame}

\section{Introduction}
\begin{frame}{Introduction}
	blabla
\end{frame}

\section{Généralités sur les Images}
\begin{frame}{Généralités sur les Images}
	blabla
\end{frame}


\section{La Biométrie}
\begin{frame}{La Biométrie}
	blabla
\end{frame}

\section{La Reconnaissance Faciale}
\begin{frame}{La Reconnaissance Faciale}
	blabla
\end{frame}


\section{Conception}
\begin{frame}{Conception}
	blabla
\end{frame}


\section{Conclusion}
\begin{frame}{Conclusion}
	blabla
\end{frame}


\section{Bibliographie}
\begin{frame}{Bibliographie}
	\begin{itemize}
	 \item \textbf{[1] A.Hamadi & Limam : « Développement d'un Système de Reconnaissance de Visages ».
Université Abdelhamid Ibn Badis ;Mostaganem. Département d'Informatique. 2008} \\
\item 
\textbf{[2] J.Landré: « Analyse Multi-Resolution Pour La Recherche Et L'indexation D'image Par
Le Contenu Dans Les Bases De Donnees Images-Application A La Base D'images
Paleontologique Trans'tyfipal ». Université de Bourgogne, Instrument et Informatique de
l'Image, 2005}\\ 
\item 
\textbf{[3] Walid Hizem : « Capteur Intelligent pour la Reconnaissance de Visage », Université
Pierre et Marie Curie-Paris6 ; Département Electronique/Informatique, 2009}\\
	\end{itemize}

\end{frame}

\begin{frame}{Bibliographie}
	\begin{itemize}
	 \item \textbf{[4] S.Z. LI and A.K. Jain : « Handbook of Face Recognition ». Springer, 2005.}\\
	\end{itemize}

\end{frame}

\begin{frame}
	Un \textbf<2,3>{texte} en gras. 
	\visible<3>{Un texte visible sur la 3\ieme{} couche}
\end{frame}

\begin{frame}{Titre (facultatif)} 
\framesubtitle{Sous titre (facultatif aussi)}
	\begin{block}{Remarque}
	Un bloc
	\end{block}
	
	\begin{alertblock}{Proposition}
	Un bloc alerte
	\end{alertblock}
	
	
	\begin{exampleblock}<2>{Exemple}
	Un bloc exemple qui est visible sur la 2\ieme{} couche : $f(x)=2x$.
	\end{exampleblock}
\end{frame}



\end{document}