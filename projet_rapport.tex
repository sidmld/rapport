\documentclass[12pt]{article}
\usepackage[francais]{babel}
\usepackage[utf8]{inputenc}
\usepackage{graphicx}
\usepackage{amsmath}
\usepackage{amsfonts}
\usepackage{amssymb}




\addtolength{\hoffset}{-1cm}
\addtolength{\textwidth}{2cm} 
\addtolength{\voffset}{-1cm}
\addtolength{\textheight}{2cm} 

\begin{document}

\begin{titlepage}
\begin{center}

\hfill
\vfill
\bigskip
\huge{ \includegraphics[width=0.3\textwidth]{logo.png} 
\includegraphics[width=0.3\textwidth]{lh.png} \\
U.F.R Sciences et Techniques \\
M2 MATIS  \\
 Rapport de stage \\  \\ \\ \\ \\
 } 
\vfill
\bigskip 
\Huge 
\bigskip %\includegraphics[width=150,height=120]{pg2.png}} 
\\ \\
\textbf{ Développement d'un site web de contrôle de gestion sous Symfony 3 } \par 
\vfill

\Large \begin{flushleft}
 Réalisé par:\\ sidi Maouloud \par
\end{flushleft}
		 
		  \begin{flushright}
		                    sous la direction de:\\  M. ould Maouloud
		                   \end{flushright}



		 
\vfill
\Large Université du Havre \par \Large VMP Consulting		
		\bigskip 
\bigskip

\Large
Juillet-Octobre 2017
\end{center}
\end{titlepage}

\section*{Remerciements}


J’adresse mes remerciements à  M. sidi ould MAOULOUD, de m’avoir accueillie pour faire mon
stage au sein de son entreprise, ainsi pour son aide et  ses  remarques pertinentes.  \\ \\
J’aimerais aussi remercier M. Laurent Amanton  et tous mes enseignants de l’université du havre
qui m’ont accompagnés 
durant l'année universitaire 2016/2017.\\ \\

Enfin  Je tiens á transmettre ma gratitude à mes
parents  pour  leur soutien.\\ \\


\newpage

\section*{Résumé}
\textbf{VMP-CONSULTING}  est un bureau spécialisé dans le contrôle de gestion pour les TPE/TPI, les PME/PMI et les collectivités territoriales. Il  accompagne et conseille les entreprises pour aller vers l'efficacité et l'efficience dans l'utilisation de leurs ressources afin d'atteindre leurs objectifs.\\
Pour mieux servir ses clients, \textbf{VMP-CONSULTING}  veut avoir un portail web pour 
mettre en œuvre un espace de travail unique pour les clients, 
 proposer aux clients un accès privilégié et personnalisé à divers services en
ligne, et 
développer des outils qui permettent aux clients de réagir avec les contenus du
portail.\\
L'objectif de ce stage est de réaliser ce site web.


\section*{Abstract}
Whatever the activity, today all companies are confronted with increased competitive pressure, technological changes ... To enable managers and operational staff to devote themselves entirely to their core business, support in managing The company is a necessity not to say an obligation. In this area, not all businesses are housed in the same category, particularly small businesses and small businesses. The lack of management tools is mainly linked to a problem of means although sometimes it can be a problem of awareness of the utility and the benefit brought by these tools.\\
\textbf{VMP-CONSULTING}  is an office specializing in management control for small and medium-sized enterprises (SMEs), SMEs and local authorities. He accompanies them and advises them to move towards efficiency and efficiency in the use of their resources in order to achieve their objectives.\\
To better serve its customers, \textbf{VMP-CONSULTING}  wants to have a web portal for
implement a unique workspace for customers,
 offer customers privileged and personalized access to various services in
line, and
develop tools that enable clients to react with the contents of the
portal.\\
The objective of this traineeship is to realize this website

\newpage
\renewcommand{\contentsname}{Sommaire}
\tableofcontents
\newpage



\section{Introduction}

\subsection{Préambule}

Quelle que soit l’activité, aujourd’hui toutes les entreprises sont confrontées à une pression concurrentielle accrue, aux évolutions technologiques...Pour permettre aux managers et aux opérationnels de se consacrer entièrement à leur cœur de métier, un accompagnement dans la gestion de l'entreprise est une nécessité pour ne pas dire une obligation. Dans ce domaine là, les entreprises ne sont pas toutes logées à la même enseigne, en particulier les TPE(Trés Petite Entreprise) et certaines PME(Petite Moyenne Entreprise). Le manque d’outils de gestion est principalement lié à un problème de moyens même si parfois il peut s’agir d’un problème de sensibilisation à l’utilité et le bénéfice apporté par ces outils.\\ \\

Le Contrôle de gestion est le garant de la bonne santé de la structure en s'assurant que les ressources sont employées efficacement.\\ 
Il intervient également pour fournir les outils qui vont servir aux décideurs pour suivre l'impact de leurs actions. \\

Dans de nombreuses entreprises, il est en charge du management du système de pilotage avec la prise en charge des tableaux de bord destinés à la direction et aux responsables opérationnels.\\ \\

Ainsi la rapidité de réaction est, plus que jamais, un facteur essentiel de l’aptitude 
d’une  entreprise  à faire  face  à la  concurrence.\\ \\

 


\subsection{VMP Consulting}


VMP-Consulting est un bureau spécialisé dans le contrôle de gestion pour les TPE/TPI(trés petite entreprise), les PME/PMI(petite moyenne entreprise) et les collectivités territoriales. Il  accompagne et  conseille  les entreprises pour aller vers l'efficacité et l'efficience dans l'utilisation de leurs ressources afin d'atteindre leurs objectifs.\\ \\


VMP Consulting accompagne les PME/PMI et TPE/TPI pour la maîtrise de leur performance, avec ses solutions, pour atteindre les objectifs de :
\begin{itemize}

\item Maîtriser les coûts.

\item Optimiser les performances.

\item Une transparence sur la gestion des ressources de  l'entreprise.

\item Le développement de la réactivité dans la prise de décisions stratégiques.
\end{itemize}
\\ \\ 


Parmi ses services :
\begin{itemize}

\item \textbf{Pilotage d'entreprise: } Des conseils en gestion et pilotage d'entreprise permettent aux managers et aux opérationnels de disposer d’un système de contrôle de la performance et d’aide à la décision.


\item \textbf{Audi: } Un audit qui permet de donner une situation précise de l'entreprise, un
diagnostic qui  permettra de développer les activités, gérer efficacement les risques et prendre les bonnes décisions stratégiques dans les meilleures conditions.


\item \textbf{Système d'information : }  Assistance à la maîtrise d’ouvrage et accompagnement du changement,
 Coordination et gestion de projets, 
 Elaboration des cahiers de charges et des spécifications fonctionnelles, 

 Audit de systèmes d’informations, et

Elaboration d’outils spécifiques.


\item \textbf{Formations: } VMP apporte toute son expérience pour des formations qui permettrons de maîtriser tous les outils nécessaires à la bonne marche de l'entreprise.
\end{itemize}
\\ \\ 



Le dirigéant de la startup a une expérience de plus de 20 ans dans divers secteurs d'activités et dans plusieurs types de structures (PME - Multinationales - Organismes d'Etat), et en 2016 il a lancé       \textbf{VMP-CONSULTING} .



 \subsection{Problématique}

L’enjeu pour une entreprise aujourd'hui est d’adresser une communication ciblée en proposant un contenu pertinent au client, par exemple, mettre en avant tous ses produits et offres complémentaires permet d’informer ses clients et de déclencher de nouvelles ventes. \\ 
Pour un bureau d’études, comme \textbf{VMP-CONSULTING}  spécialisé dans la maîtrise des performances des entreprises, la mise en place d'un site internet,  permet de  Capitaliser les informations et les savoir-faire,
    simplifier la recherche d’informations,
    centraliser l’ensemble des données en un seul accès et
    fédérer les collaborateurs et les utilisateurs autour de l’entreprise.
\\ \\

Un portail web est une plate-forme collaborative dont la fonction première est de proposer aux internautes des ressources et services numériques en rapport avec un thème, un domaine d’intérêt et dédié à chaque communauté particulière (les collaborateurs, les partenaires, les clients ou encore les fournisseurs …). \\
Il s’agit d’un espace de travail unique, personnalisé et sécurisé avec des droits d’accès par utilisateur.\\ \\


Ce future site web de \textbf{VMP-CONSULTING}, dédié aux entreprises clientes, sera un outil important
 de gestion des entreprises en ligne et qui pourra répondre à certains besoins du bureau d'études:
\begin{itemize}

\item  Dynamiser la collaboration avec les interlocuteurs internes et externes
 \item   Permettre la transmission des connaissances pour une organisation plus productive et performante
 \item     Améliorer le suivi des prestations réalisées par les consultants techniques : suivre en temps réel les interventions chez les clients afin d’accélérer la prise en charge directement depuis le terrain, disposer d’indicateurs et de statistiques pour détecter les problèmes récurrents.
 
 \item   Renforcer  l'offre auprès des clients, les fidéliser avec de nouveaux services
 \item   Optimiser certaines tâches back-office : accès aux données et à l’historique des  clients, avancement des taches, ...
\item    Répondre à de nouvelles exigences et garder une longueur d’avance.

\end{itemize}

\\ \\

A l'aide d'un portail web  \textbf{VMP-CONSULTING} pourra atteindre ses objectifs :

\begin{itemize}

\item Améliorer la productivité des services de l'entreprise.
\item Fidéliser les clients.
\item Se différencier par rapport à la concurrence en proposant un panel de services à
valeur ajoutée.
\item Conquérir de nouveaux clients.
\end{itemize}



\subsection{Environnement et Missions}

La mission était la prise en charge totale du projet de mise en place d'un portail internet,
j'avais en charge toutes les phases du projet de la rédaction de cahier de charges jusqu'à la phase de test en 
passant par les spécifications personnelles, les choix techniques, la modélisation et conception.\\
Parmi les difficultés rencontrées est le manque de ressources en développement de VMP-CONSUTLING ce qui a nécessité pour moi 
de passer plus de temps dans la recherche des outils de développement.
Mes missions en tant que le seul développeur du projet sont : \\
\begin{itemize}
\item Rédaction  du cahier des charges.
\item Analyse des données et proposition des solutions.
\item Conception  du site internet .  
\item Modélisation de la base de données.
\item Développement et Implémentation .
\item test et déploiement du portail web.
\end{itemize}
\\ \\
On parle souvent de cycles de vie, qui ont pour but d’organiser ces étapes de différentes manières en fonction d’un certain nombre de critères relatifs au projet de développement.\\
Le cycle de vie en cascade a été utilisé, puisque 
les besoins ou exigences n’évolueront pas en cours de projet, 
tous les besoins ont été fixés et détaillés dans le cahier des charges.
En pratique, une étape ne démarre que si la précédente ait été validée par le client.

\\
\begin{center}
\begin{figure}[htp]
  \centering
  \includegraphics[width=12cm]{p.png}
  \caption{Cycle de vie en cascade, cours gestion des projets d'openclassrooms.com.}
  \label{fig:une-autre-image}
\end{figure}
 
\end{center}
\\ \\



\newpage

\section{ Cahier des charges et spécifications fonctionnels}

\subsection{ Cahier des charges}
Un cahier des charges concrétise une demande et constitue la première description complète du projet web à réaliser. Il tient compte des besoins de ceux qui ont fait l’appel à projet et de ceux qui s’engagent à le développer.\\


A partir de la problématique définit par \textbf{VMP-CONSULTING} le début du stage était d'abord 
la rédaction du cahier des charges.\\ \\
Dans la première section du cahier des charges, j'ai essayé de donner une vision globale du projet au client, 
en rappelant les objectifs principaux de ce projet.




\subsubsection{Objectifs}

Les objectifs de ce projet sont pour l'entreprise \textbf{VMP-CONSULTING} : 
\begin{itemize}
\item \textbf{  Mettre en œuvre un espace de travail unique pour les clients}: qui garantit 
l'accessibilité des données économiques et financières en temps réel en  responsive design.

\item \textbf{ Proposer aux clients un accès privilégié, personnalisé et sécurisé  à divers services en
ligne}:  par exemple, Suivre en temps réel l'ensemble des tableaux des bords et d'indicateurs des performance. 
 
\item \textbf{ Développer des outils qui permettent aux clients de réagir avec les contenus du
portail}: par exemple, Faire des prévisionnels, saisir des hypothèses d'évolutions des indicateurs de gestion  et calculer des projections de résultat.


\end{itemize}

\\



\subsubsection{Étude de l’existence}

Lors de la réalisation du cahier des charges, je doit prendre en compte un grand nombre de besoins :
\begin{itemize}



\item Le projet doit tenir compte de contraintes globales (accessibilité etc.).

\item Le projet doit correspondre à la création graphique validée par le client.

 \item Le projet doit tenir compte du budget défini en amont.

 \item   Le projet s’adapte très souvent sur un existant (système d’information etc.).

\end{itemize} \\
\\

Actuellement \textbf{VMP-CONSULTING} utilise  des outils performants et facile à manipuler 
pour faire le contrôle de gestion de ses clients.\\
Les analyses et les calculs du futur système de gestion en ligne sont basés
sur des fichiers existants en Excel.\\

Les données et les résultats qui contiennent ces fichiers ont été étudié attentivement afin de les transformer en  une base de données interactive qui peut être consultée en ligne en toue sécurité, et ses données peuvent être visualisées en temps réel.

\\
\\



 




\subsection{ Spécifications fonctionnelles}


Le cahier des charges fonctionnel et technique est un document contractuel. Il résume le périmètre fonctionnel et technique du projet attendu par le client sur lequel le développeur s’engage de le réaliser.\\
La 2e section du cahier des charges consiste à rédiger les spécifications fonctionnelles et techniques du projet. C’est ici que j'ai  présenté en détail le dispositif du projet. \\

On va voir par la suite  des exemples où j'ai résumé quelques Spécifications fonctionnelles.

\subsubsection{L'accès au portail}
 
\begin{description}

\item[Fonctionnalité :] Page d'accueil.

\item[Description :] Une page d’accueil qui s'ouvre par
défaut en accédant au portail.

\item[Résultat :]  Tout utilisateur web peut accéder au site  par internet et visiter sa  page d'accueil qui contient le plan du portail et les informations de l'entreprise.

\end{description}


\subsubsection{ Création d’espace/entreprise}


\begin{description}

\item[Fonctionnalité :] Création de compte.

\item[Description :]  Un espace sécurisé et personnalisé permettant la gestion des données personnelles.

\item[Résultat :]  Toute entreprise client peut s'inscrire
au site de l'entreprise et avoir un compte dédié qui contient ses informations et ses données.

\end{description}

La création d'un espace Entreprise passe par les étapes suivantes:
\begin{itemize}


\item Demander à l'utilisateur d'entrer son adresse mail dans une zone de texte.
\item Demander à l'utilisateur d'entrer un pseudo dans une zone de texte.
Le pseudo devra contenir entre 3 et 25 caractères.
\item
Demander à l'utilisateur d'entrer un mot de passe qui devra compter entre 6 et 30 caractères, là encore dans une zone de
texte.
\item
Demander une confirmation du mot de passe.
Une fois les deux mots de passe entrés, si les deux mots de passe sont les
mêmes, on continue l'inscription.
Si ce n'est pas le cas, on redirige l'utilisateur vers une page d'erreur et on lui fait reprendre l'inscription du départ
\item
On l'envoie vers une seconde page de l'inscription, en le faisant cliquer sur un lien.

\end{itemize}



\subsubsection{Connexion}
\begin{description}


\item[Fonctionnalité :] Connexion .

\item[Description :]  accès au compte personnel par un pseudo et un mot de passe.

\item[Résultat :]  Chaque utilisateur peut accéder à
son espace personnel en toute sécurité.
\end{description}

\subsubsection{ Consultation des données}
\begin{description}

\item[Fonctionnalité :] Consultation .

\item[Description :]  Consulter les données et les informations du de l'entreprise.

\item[Résultat :]  Chaque utilisateur peut consulter toutes ses informations personnelles  et visualiser ses 
données.
\end{description}

\textbf{Consultation des indicateurs}:
\begin{itemize}
\item  Chiffre d’affaires
\item Couts
\item Marge
\item Résultat
\item Historiques
\end{itemize}


\subsubsection{ Importation/Exportation}
\begin{description}

\item[Fonctionnalité :] Importation/Exportation.

\item[Description :]  Importer et exporter des données et fichiers (xsl, csv )

\item[Résultat :]  l'utilisateur peut importer des fichiers Excel et télécharger les données sous forme
 des  fichiers csv.
\end{description}

\subsubsection{ Simulations }
\begin{description}

\item[Fonctionnalité :] Simulations.

\item[Description :]  permet de faire des hypothèses de projection de résultat.

\item[Résultat :]  Un tableau de bord interactif.
\end{description}

\subsubsection{Gestions des Editions}
\begin{itemize}

\item Compte de résultat 
 
\item Statistiques

\item Tableaux de bord
\end{itemize}

\subsubsection{Alert et Info}
\begin{description}

\item[Fonctionnalité :] Alert et Info.

\item[Description :]  Envoi des messages d'alertes et des e-mails d'informations aux utilisateurs.

\item[Résultat :]  En cas de création de compte, de changement de mot de passe et d'informations importantes, 
l'utilisateur reçoit des messages, à son adresse e-mail, de confirmation de versification et d'informations.
\end{description}



\newpage 

\section{ Choix techniques}


\subsection{Système d'exploitation}
Après la rédaction du cahier des charges  on passe à la recherches des solutions pour les besoins définis dans ce cahier. \\
A noter que j'avais toute la liberté de choisir, rien n'était exigé par l'entreprise, mais à condition que 
le choix soit bien justifié.\\ \\

Le système d'exploitation choisi est Linux avec sa distribution \textbf{Ubuntu} , il est open source, sécurisé et son environnement est bien adapté au développement du projet. 

\subsection{Architectures}
Un des plus célèbres design patterns s'appelle MVC, qui signifie Modèle - Vue - Contrôleur. C'est celui que je me suis basé pour la construction de l’architecture du site web.

Le pattern MVC permet de bien organiser son code source. Il permet de savoir quels fichiers créer, mais surtout à définir leur rôle. Le but de MVC est justement de séparer la logique du code en trois parties(Modèle - Vue - Contrôleur).\\

\begin{center}
\begin{figure}[htp]
  \centering
  \includegraphics[width=12cm]{mvc.png}
  \caption{Architecture mvc, cours de php d'openclassrooms.com.}
  \label{fig:une-autre-image}
\end{figure}

\end{center}


\subsection{Méthode de Programmation }

Les principales solutions existantes pour réaliser ce type de site Web sont :
\begin{itemize}
\item le codage « from scratch », c'est à dire en partant de zéro en utilisant un langage de programmation de A à Z.
\item l'utilisation d'un CMS qui  nous évite les lignes de codes .
\item l'utilisation d'un framework basée sur un langage de programmation.
\end{itemize} \\
La solution from scratch a été abandonnée, car elle est trop lourde et longue à
développer, surtout quand il s'agit d'un stage de quelques mois avec beaucoup des choses à réaliser.\\ 
Ainsi la solution CMS n'a pas été retenue puisqu'elle n'est pas adapté pour le genre de base de données interactive que 
je souhaite mettre en œuvre, avec la lenteur d'accès aux bases de données qui est visible surtout à l'affichage des pages
est un grand inconvénient d'un CMS.\\ \\

En revanche, l'utilisation d'un framework implique le développement sur-mesure de
tous les éléments du site à l'aide de fonctions relativement simples. L'apprentissage au
développement avec un framework apparaît plus simple qu'avec un CMS.\\

C'est pourquoi, le choix est d'utiliser un framework pour le développement du site web.\\

Mais le problème de choix ne s’arrête pas ici, il reste de préciser le choix du langage web et son framework.
 
\subsection{Langage web coté serveur }
Le langage de programmation utilisé va beaucoup influer sur le projet et la manière dont celui ci sera développé, en fonction des avantages et des inconvénients du langage.\\
On parle ici du langage back-end du site web, puisque le développement back-end est nécessaire pour bien gérer 
la bases de donnés, les  utilisateurs et les résultats à envoyer, en basant sur l'architecture MVC.\\ \\
Le choix du langage web est une autre étude qui a pris pas mal de temps en cherchant les différents langages, leurs  avantage et inconvénients. Les recherches étaient concentrés sur les langages suivants :
\begin{itemize}
\item PHP
\item Nodejs (JavaScript back-end)
\item Java (JavaEE)

\end{itemize}

Les autres langages ont été abandonné immédiatement par le motif de manque d'une base de connaissance permettant
l'apprentissage rapide du langage.\\ \\

Le choix du langage s’est finalement porté sur \textbf{PHP} avec sa version 5, qui 
 est un langage de script exécuté côté serveur. Langage qui permet une interaction avec l’utilisateur. Technologie permettant la création de pages web au contenu dynamique.\\

 En effet, il s’agit d’un langage facile
d’apprentissage, accessible sur la plupart des systèmes d’exploitations et très populaire sur le
web, ce qui permet un meilleur support et une meilleure maintenance. De plus, il s’agit d’un
langage déjà éprouvé depuis plusieurs années et donc assez robuste pour répondre aux
besoins de l’entreprise, qui veut s’appuyer sur des technologies matures et fiables pour
fonctionner de manière optimale. Enfin, il est assez facile d’apprentissage, ce qui permettra à
de futurs développeurs de maintenir ou de faire évoluer rapidement l’application.\\  \\




\subsection{Framework }
Un framework ou kit de développement est un espace de travail modulaire, c'est à dire
une suite d'outils et de bibliothèques qui facilitent et accélèrent le développement d'un
logiciel. Il contient toutes les fonctions de base utiles au développement d'un type de
programme, et permet donc de ne pas avoir besoin de réécrire les mêmes fonctions à
chaque programme créé. Il en existe dans tous les langages de programmation.\\ \\
J'ai testé et étudié les plus connus pour trouver le plus adapté:
\begin{itemize}
\item Symfony 3
\item Zend Framework 2
\item Laravel
\end{itemize} \\

 J'ai décidé d'utiliser Symfony 3 (précisément sa version 3.3), qui permet de  rendre le PHP beaucoup plus
confortable, l'AJAX plus abordable, l'optimisation du référencement (url rewriting) plus
simple.\\
Symphony est un Framework PHP qui a été lancé en 2005. Il est aujourd’hui stable et reconnu.
Il est également orienté objet, respecte le modèle MVC et est développé sous licence MIT.
C’est un Framework très utilisé et reconnu internationalement. Il a été développé par la société
SensioLabs qui l’utilise et le maintien régulièrement.
Il est considéré comme un ensemble d’outils rassemblant des composants préfabriqués,
rapides et faciles à utiliser.\\
Un des avantages de Symfony est de proposer une évolutivité et une maintenance efficace en
permettant à d’autres développeurs de prendre en main rapidement le projet sans avoir
participé à son élaboration. Il existe également un nombre important de ressources sur le web
pour rendre la maintenance encore plus facile. Enfin, il est très flexible car il permet de n’utiliser
que certains de ces modules sans forcément avoir à utiliser tout le Framework. Laravel
possède beaucoup de composants issus du Symfony.


\subsection{Bases de données}
 Le choix de la base de donnée  a besoin  d'un temps de recherche et de comparaison entre les deux principaux modèles, le 
  \textbf{Relationnel} et le \textbf{NoSql} . \\
   \\
Et enfin j'ai choisi le modèle relationnel et son SGBDR MySql.\\

MySQL est le plus connu et utilisé des SGBD. Il repose sur le modèle relationnel, des tables ont des enregistrements , et ces tables peuvent avoir des relations.\\  Ceci a l’avantage de pouvoir lier très facilement des enregistrements d’une table à l’autre.\\  De plus, lors des enregistrements, les transactions sont soumises aux contraintes ACID (atomicité, cohérence, isolation et durabilité), ce qui signifie qu’un enregistrement incomplet ou incorrect ne sera pas enregistré en base. \\  MySQL permet ainsi de facilement structurer les informations et de les réutiliser avec aisance. Finalement, MySQL est un système où l’intégrité des enregistrements sont prises en charge par le logiciel et le risque d’erreurs est donc peu élevé.


  
   

\subsection{Langages web coté client}

L'effet de choisir ce n'est pas toujours simple surtout dans le cas où les concurrents sont très proches.
Mais heureusement dans la partie front-end (coté client) il y a moins de choix.\\ \\

L'utilisation des langages suivants est nécessaire et parfois est indispensable  :
\begin{itemize}
\item \textbf{HTML5:} L’Hypertext Markup Language(HTML), est le format de données conçu pour gérer
et organiser le contenu d'une page web. C’est un langage de balisage qui
permet d’écrire de l’hypertexte, d’où son nom. C'est un langage de description
de données, et non un langage de programmation. Je l’ai utilisé pour créer la
partie statique du site web.

\item \textbf{CSS3:} Cascading Style Sheets : feuilles de style en cascade est un langage informatique
qui sert à décrire la présentation des documents HTML (et XML). Les standards
définissant les CSS sont publiés par le World Wide Web Consortium (W3C). Introduit au
milieu des années 1990, le CSS devient couramment utilisé dans la conception de sites
web et bien pris en charge par les navigateurs web.

\item \textbf{JAVASCRIPT:} Javascript est un langage de programmation de type script, non compilé, orienté
objet, principalement utilisé dans les pages Web. C’est un langage exécuté  côté client, c'est-à-dire par le navigateur de l’utilisateur. Il a pour but de dynamiser les
sites Internet.
\end{itemize}

\subsection{Technologies et Outils}

On peut résumer la suite  des frameworks et des  bibliothèques utilisés  par la liste suivante :
\begin{itemize}

\item \textbf{JQUERY:} En JavaScript j’ai utilisé plus particulièrement ​
jQuery, qui est une bibliothèque
JavaScript libre qui porte sur l'interaction entre JavaScript
(comprenant Ajax) et HTML, et a pour but de simplifier des
commandes communes de JavaScript. C’est avec cette technologie que j’ai réalisé la partie dynamique coté client du site web.

\item \textbf{AJAX:} AJAX est l'acronyme d'Asynchronous JavaScript And XML, autrement dit JavaScript Et XML Asynchrones.
L'idée  est de faire communiquer une page Web avec un serveur Web sans occasionner le rechargement de la page. 

\item \textbf{Bootstrap:} kit CSS créé par les développeurs de Twitter, est devenu en peu de temps le framework CSS de référence. Il permet de  construire rapidement et facilement des sites web esthétiques et responsives. Bootstrap offre aussi des plugins jQuery de qualité pour enrichir les pages.

\item \textbf{WEBSOCKET:} WebSocket est une alternative à Ajax plus simple à mettre en oeuvre coté client, mais avec une compatibilité limitée aux navigateurs récents.

\item \textbf{TWIG:} Dans Symfony le PHP et le HTML sont entièrement séparer, le HTML est inséré
dans les fichiers Twig. Twig est un moteur de template PHP
directement intégré dans Symfony3 et créé lui aussi par Sensio. Très
puissant, Twig permettra de gérer de l’héritage entre templates et
layout, séparer les couches de présentation et couche métiers.

\item \textbf{Doctrine: } l'ORM par défaut de Symfony. L'objectif d'un ORM (pour Object-Relation Mapper, soit en français « lien objet-relation ») est simple : se charger de l'enregistrement des données en  faisant oublier qu'il n'y a pas  une base de données.

\end{itemize}
\\ 
\\

\subsubsection{Outils}

\begin{itemize}


\item \textbf{GIT:} un outil qui va  permettre de versionner le code source, c'est-à-dire gérer les versions du code au fur et à mesure. j'ai l'utilisé pour pour héberger le code source  en github, en cas de panne locale le
code est facile à récupérer.

\item  \textbf{Apache2:} est un serveur HTTP créé et maintenu au sein de la fondation Apache. C'est le serveur HTTP le plus populaire du World Wide Web. il me permet de tester le site localement.

\item  \textbf{Netbeans:} est un environnement de développement intégré (EDI) open source. Il a
été créé par Sun en juin 2000. NetBeans permet de supporter de nombreux
langages dont PHP, HTML, Twig, Javascript et YML. C’est sous cet environnement que j’ai programmé l’intégralité du portail.


\item \textbf{draw.io:} c'est un outil en ligne permettant de faire des design des maquettes, templates et l’architecture du site web.

\item \textbf{Umbrello:} c'est un logiciel linux open source permettant de faire des diagrammes UML.

\item \textbf{Mysql-server:} le serveur de la base de données MySql

\item \textbf{PhpMyAdmin: } Permettant de visualiser le données au navigateur.

\item \textbf{FileZila:} Transfert ftp.

\item \textbf{TexMaker} Éditeur latex pour la rédaction du cahier de charges, des comptes rendu et des rapports.
\end{itemize}



\newpage

\section{Conception et Modélisation  }

\subsection{Conception}

Dans la phase d’analyse, le diagramme de classes représente les entités  manipulées par les utilisateurs.\\
Dans la phase de conception, il représente la structure objet d’un développement orienté objet.\\
\\

\begin{center}
\begin{figure}[htp]
  \centering
  \includegraphics[width=12cm]{dc.png}
  \caption{Diagramme de classe.}
  \label{fig:une-autre-image}
\end{figure}

\end{center}



Pour simplifier les choses, on peut résumer les principales fonctionnalités, déjà précisées dans le cahier des charges, par un diagramme UML de cas d'utilisation , qui  représente les fonctionnalités  nécessaires aux utilisateurs.\\

\\
\\

\begin{center}
\begin{figure}[htp]
  \centering
  \includegraphics[width=12cm]{cd2.png}
  \caption{Cas d'utilisation.}
  \label{fig:une-autre-image}
\end{figure}

\end{center}


\\
\\

Toutes les fonctionnalités détaillées dans le cahier des charges, devront être dans les bonnes places dans l'interface web du site internet.\\
Ainsi j'ai procédé par la suite à modéliser l'ensemble des pages du portail afin de mettre chaque 
fonctionnalité à sa place.\\




\begin{center}
\begin{figure}[htp]
  \centering
  \includegraphics[width=12cm]{p3.png}
  \caption{Les templates principales du site  .}
  \label{fig:une-autre-image}
\end{figure}

\end{center}

\\ \\
J'ai choisi de minimiser le nombre des pages  , en utilisant des pages
\textbf{scrolling} qui contiennent plusieurs sections.







\subsection{Modélisation}
\subsubsection{Modélisations des données}




La modélisation de la base de données est également une tâche très importante car il s’agit
du cœur du site web à réaliser. \\ 


\\ \\

\begin{center}
\begin{figure}[htp]
  \centering
  \includegraphics[width=12cm]{bd1.png}
  \caption{Les tables de la bases de données.}
  \label{fig:une-autre-image}
\end{figure}

\end{center}
\\ \\

La figure ci-dessous représente la base de données réalisée, à noter que toutes les attributs des entités  ne s'affichent pas dans la figure .\\ Nous pouvons constater que quasiment toutes les tables sont reliées à la table  \textit{societe}  . Cela nous indique que les requêtes effectuées sur la base de données concernent essentiellement les sociétés, qui sont les clients de \textbf{VMP-CONSULTING}.\\
La base de données elle-même comportant un nombre assez important de tables.\\
 Il est possible que
ces attributs soient trop nombreux et doivent être séparés pour être mises dans des nouvelles
tables.\\



\subsubsection{Maquette et Responsive design}

Une maquette permet de vérifier avec le client tous les aspects visuels du site, de la mise en page générale aux détails tels que les couleurs, les polices , les images, les icônes, les logos… etc.\\
Elle est l’équivalent du site web en version imprimée. La création de la maquette vient suite à l’étude des besoins du client et l’élaboration d’un cahier des charges.\\

 \\
\\


\begin{center}
\begin{figure}[htp]
  \centering
  \includegraphics[width=12cm]{mod2.png}
  \caption{Maquette de la page d'accueil.}
  \label{fig:une-autre-image}
\end{figure}

\end{center}

\\ \\


Le responsive design est important dans le processus de création d’un site web.  C'est parce que internet est maintenant présent sur tous les écrans, que ce n'est plus possible de concevoir un site web sans penser au responsive design.\\
Un temps de conception d'une maquette responsive est nécessaire afin d'avoir une vision globale et précise de site responsive.\\ \\
Le responsive design est une fonctionnalité importante du cahier des charges.
Et la  de la maquette ci-dessus  a été traduite en \textbf{html5} et \textbf{CSS3}, pour devenir une  interface web responsive
 à l'aide de \textbf{Boostrap}.\\
Cette interface a été  dynamisé par \textbf{javascript} et sa librairie \textbf{JQuery}.
 
 \\
 \\

\begin{center}
\begin{figure}[htp]
  \centering
  \includegraphics[width=12cm]{design1.png}
  \caption{Réalisation de la maquette.}
  \label{fig:une-autre-image}
\end{figure}
\end{center}

\\ \\ 


\newpage

\section{Développement et Implémentation}

\subsection{Installation et Configuration}
Dans cette partie, on retrouve  les étapes à suivre pour amorcer un  projet Symfony 3.3 sous Linux Ubuntu .\\
 Cela implique l'installation d'un serveur Apache, d'un système de gestion de base de données MySQL, du langage de programmation PHP, de l'administration avec PhpMyAdmin et l'outil de développement "framework" Symfony.\\ 


\subsubsection{Environnement}
J'ai effectué, l'installation de paquets, par l'intermédiaire d'un terminal. Une série de commandes a été effectuée afin de les installer sur la machine.
\\

\fbox{ \textit{\textbf{sudo apt-get  install apache2}} }\\

\fbox{ \textit{\textbf{sudo apt-get  install  php5}} }\\

\fbox{ \textit{\textbf{sudo apt-get  install libapache2-mod-php5}} }\\

\fbox{ \textit{\textbf{sudo apt-get  install  mysql-server}} }\\

\fbox{ \textit{\textbf{sudo apt-get  install  php5-mysql}} }\\

\fbox{ \textit{\textbf{sudo apt-get  install php-apc}} }\\

\fbox{ \textit{\textbf{sudo apt-get  install php5-intl}} }\\

\fbox{ \textit{\textbf{sudo apt-get  install  phpmyadmin}} }\\

   
     
\\
\\
Symfony est le Framework PHP utilisé, crée par SensioLabs basé sur une
architecture MVC( Modéle, vue et contrôleur ). Ce framework a été choisi
en raison des motifs expliqués dans la partie \textbf{Framework}.\\
La version utilisé est la dernière version sortie à l’écriture de ce rapport \textbf{symfony 3.3}.
\\ 
J'ai besoin ici de la version 5.5 au minimum du php pour le bon fonctionnement de \textbf{symfony 3}.\\
Il existe plusieurs méthodes pour obtenir Symfony 3. J'ai utilisé la méthode de le Symfony Installer. Il s'agit d'un petit fichier PHP (un package PHAR) à télécharger puis exécuter sur le terminale.\\

Pour le télécharger, il suffit d'aller vers l'adresse  symfony.com/installer. Cela  télécharger un fichier \textbf{symfony.phar} , que sera déplacé dans le répertoire web habituel. \\
Ce fichier permet d'exécuter plusieurs commandes, mais la seule qui nous intéresse pour l'instant est la commande "\textbf{new}", qui permet de créer un nouveau projet Symfony en partant de zéro.\\
La commande suivante créera un nouveau projet  "\textbf{Symfony}":\\
\fbox{ \textit{\textbf{php symfony.phar new Symfony 3.3.2}} }\\
 \\


\subsubsection{Symfony 3}
Dans cette partie, je vais présenter comment est organisé Symfony . Le but est e d'avoir une vision globale du processus d'exécution d'une page sous Symfony.\\
\begin{description}
\item[App :]  c'est  pour séparer le code source, qui fait la logique du site, de sa configuration. Ce sont des fichiers qui concernent l'entièreté du site, contrairement aux fichiers de code source  découpés par fonctionnalité du site.
\item[Bin :] ce répertoire contient tous les exécutable du développement.
\item[Src :] le répertoire dans lequel on mettra le code source, c'est ici que l'on passera le plus de temps. Dans ce répertoire, j'ai organisé le code en \textbf{bundles}.
\item[Tests :] contient tous les tests de  l'application. Les tests étant un pan entier du développement et indépendant de Symfony.
\item[Var :] il contient tout ce que Symfony va écrire durant son process : les logs, le cache, et d'autres fichiers nécessaires à son bon fonctionnement.
\item[Vendor :] il contient toutes les bibliothèques externes à notre application. Dans ces bibliothèques externes, il y a Doctrine, Twig, SwiftMailer, etc.
\item[Web :] il contient tous les fichiers destinés aux visiteurs : images, fichiers CSS et JavaScript, etc. 

\end{description} \\ \\

Quand on parle de symfony le le terme \textbf{bundle} est souvant évoqué.\\
Un bundle est une brique de  l'application web. Symfony utilise ce concept  qui consiste à regrouper dans un même endroit,  tout ce qui concerne une même fonctionnalité.
Il regroupe les contrôleurs, les modèles, les vues, les fichiers CSS et JavaScript, etc. Tout ce qui concerne directement la fonctionnalité  du site web.\\
J'ai utilisé dans ce projet des bundles importants qui m'ont facilité le développement du site web.\\ \\

Pour installer un bundle extérieur, un \textbf{composer} est indispensable.\\
\textbf{Composer} est un outil pour gérer les dépendances en PHP. Les dépendances, dans un projet, ce sont toutes les bibliothèques dont le projet dépend pour fonctionner. Par exemple, ce  projet utilise la bibliothèque SwiftMailer pour envoyer des e-mails, il « dépend » donc de SwiftMailer. Autrement dit, SwiftMailer est une dépendance du projet.\\



\subsection{Les entités }

\subsubsection{L'ORM Doctrine}
L'objectif d'un ORM  est de se charger de l'enregistrement des  données en  faisant oublier qu'il y n'a pas de base de données. \\
IL s'occupant de tout, Je n'ai pas écrit de requêtes, ni créer de tables via phpMyAdmin. Dans le code PHP, j'ai  fait appel à Doctrine2, l'ORM par défaut de Symfony, pour faire tout cela.\\ \\

Il  faut créer une base de données avant de lancer les commandes Doctrine.\\
 En effet, pour pouvoir lancer les commandes provenant de DoctrineBundle, la librairie vérifie la connexion à la base de données configurée dans le fichier \textbf{ app/config/parameters.yml}. \\
 Ce fichier doit être bien configurer, le nom de la base de données, le mot de passe, le port, ... etc.\\
 Et pour créer un base de donnée il suffit de lancer la commande :\\
 \fbox{ \textit{\textbf{php bin/console doctrine:database:create}} }\\
 \\
 Et pour la création et la la génération d'une entité, on lance la commande suivante et on suit le guide :\\
 \fbox{ \textit{\textbf{php bin/console doctrine:generate:entity}} }\\
 
\\ 

La commande suivante permet de générer les tables à l'intérieur de la base de données :\\
\fbox{\textbf{php bin/console doctrine:schema:update --dump-sql}}\\
Et pour exécuter concrètement les requête sql :\\
\fbox{\textbf{php bin/console doctrine:schema:update --force}}\\ \\

Pour enregistrer les entités on utilise le service \textbf{EntityManager}. 
 C'est ce service qui permet de préciser à  \textbf{Doctrine} les objets à persister .\\
 
\\

Et pour la récupération des données, on utilise les repositories .\\
Ce sont donc ces repositories qui nous permettront de récupérer nos entités. \\


\subsubsection{Les Relations}
Il y a plusieurs façons de lier des entités entre elles. \\
 Soit avec  des relations \textbf{OneToOne},\textbf{OneToMany} ou \textbf{ManyToMany}. \\
 Les notions de propriétaire et d'inverse, et de  unidirectionnalité et de bidirectionnalité sont  importantes dans les relation Doctrine. Dans une relation entre deux entités, il y a toujours une entité dite propriétaire, et une dite inverse.\\ 
 Et aussi  une relation peut être à sens unique ou à double sens.\\ \\





\subsection{Administration}



 Pour faire le back office d'un  site web en symfony, il existe  le CRUD generator inclu dans symfony, mais dans ce cas pas mal de code à faire et surtout beaucoup de CSS.\\
 C'est pour cela que j'ai utilisé le bundle \textbf{EasyAdminBundle} pour le développement du back-office.\\
 Ce bundle m'a facilité la construction des taches administration et il permet de développer toutes les 
 fonctionnalités concernant  l'administrateur du site.\\ \\
 
 EasyAdmin bundle est très simple à installer et configurer. Sa documentation est très bien faite .
\\
Pour l'installer, on lance la commande \textbf{composer} :\\

\fbox{ \textbf{composer require javiereguiluz/easyadmin-bundle} }  \\

Après on modifie  le fichier\textbf{ app/AppKernel.php}, et la fonction \textbf{registerBundles()} .


Puis on fait  la configuration  dans le fichier \textbf{app/config/config.yml}, et on rajoute ses routes dans le fichier \textbf{app/config/routing.yml}.\\

Et enfin on modifie le bundle  afin qu'il soit adapté à notre projet.\\ \\

\begin{center}
\begin{figure}[htp]
  \centering
  \includegraphics[width=12cm]{v8.png}
  \caption{Interface administrateur.}
  \label{fig:une-autre-image}
\end{figure} \\ \\
\end{center}


\subsection{Sécurité}
La sécurité sous Symfony est très poussée, on peut la contrôler  finement et facilement. 
Pour atteindre ce but, Symfony a bien séparé deux mécanismes différents : l'\textbf{authentification} et l'\textbf{autorisation}.
\\
\\
L'authentification est le processus qui va définir qui nous sommes, en tant que visiteur,  soit nous ne nous sommes pas identifiés sur le site et nous somme un anonyme, soit nous nous sommes identifiés via le formulaire d'identification  et nous sommes un membre du site. C'est ce que la procédure d'authentification va déterminer. Ce qui gère l'authentification dans Symfony s'appelle un \textbf{firewall}.\\
\\

Et l'autorisation est le processus qui va déterminer si nous avons le droit d'accéder à la page demandée. \\
Il agit donc après le firewall. Ce qui gère l'autorisation dans Symfony s'appelle \textbf{l'access control}.\\
\\
\begin{center}
\begin{figure}[htp]
  \centering
  \includegraphics[width=12cm]{s.png}
  \caption{Exemple de sécurité, cours Symfony d'openclassrooms.com.}
  \label{fig:une-autre-image}
\end{figure} \\ \\
\end{center}
\\
\\
Lorsqu'un utilisateur tente d'accéder à une ressource protégée :
\begin{enumerate}


\item    Un utilisateur veut accéder à une ressource protégée 

 \item    Le firewall redirige l'utilisateur au formulaire de connexion 

  \item   L'utilisateur soumet ses informations d'identification 

 \item    Le firewall authentifie l'utilisateur 

 \item    L'utilisateur authentifié renvoie la requête initiale 

 \item    Le contrôle d'accès vérifie les droits de l'utilisateur, et autorise ou non l'accès à la ressource protégée.
\end{enumerate} \\ 
\\
Pour bien sécuriser son site web  sous symfony, il faut bien configurer le fichier \textbf{security.yml}, situé dans le répertoire \textbf{app/config}.\\
 La sécurité fait intervenir de nombreux acteurs et demande pas mal de travail de mise en place. \\


\subsection{Front-end}

\subsubsection{Les templates}
PHP lui-même est un moteur de template, mais  il n'a pas évolué comme celui de ces dernières années. En fait, il ne supporte pas beaucoup de fonctionnalités que les moteurs modernes devraient avoir de nos jours.\\
C'est pourquoi j'ai utilisé \textbf{Twig}, qui est 
un moteur de template moderne pour PHP, pour le développement des interfaces web.\\ \\

\textbf{Twig} compile des modèles vers le code PHP optimisé. Les frais généraux par rapport au code PHP régulier ont été réduits au minimum. Il  a une syntaxe très concise, qui rend les modèles plus lisibles, et il supporte tout ce dont nous avons besoin pour créer facilement des modèles puissants: héritage multiple, blocs, sorties automatiques, etc...\\ \\

L'un des objectifs de \textbf{Twig} est d'être le plus rapide possible.\\\\


\begin{center}
\begin{figure}[htp]
  \centering
  \includegraphics[width=12cm]{v2.png}
  \caption{Section de présentation de l'entreprise.}
  \label{fig:une-autre-image}
\end{figure}
\end{center}

\subsubsection{Gestion des utilisateurs}
Dans Symfony, un utilisateur est un objet qui implémente l'interface \textbf{UserInterface}.\\

\textbf{FOSUserBundle} est un bundle qui s'occupe de gestion des utilisateurs, il est très utilisé par la communauté Symfony et répond à un besoin  basique d'un site Internet : l'authentification des membres.\\ \\

Pour son installation, \textbf{composer} s'occupe de ça par la commande:\\

\fbox{\textbf{ php composer.phar require friendsofsymfony/user-bundle} }
\\ \\
Puis on active le bundle dans le fichier \textbf{app/AppKernel.php}.\\
\textbf{FOSUserBundle} est un bundle générique évidemment, car il doit pouvoir s'adapter à tout type d'utilisateur de n'importe quel site internet.\\
Il reste à personnaliser le bundle que l'on vient d'installer, 
afin de faire correspondre le bundle à nos besoins. \\
\\
\begin{center}
\begin{figure}[htp]
  \centering
  \includegraphics[width=12cm]{v5.png}
  \caption{Page de connexion.}
  \label{fig:une-autre-image}
\end{figure}
\end{center}

\subsubsection{Tableau de bord}
 
Un tableau de bord de données est un outil de gestion de l'information qui suit, analyse et affiche visuellement les indicateurs clés de performance  pour surveiller la santé d'une entreprise, d'un service ou d'un processus spécifique. \\
Ils sont personnalisables pour répondre aux besoins spécifiques  d'une entreprise.\\
\\


 Dans les coulisses, un tableau de bord se connecte à une 
 base des données, des services et API.\\
  Un tableau de bord de données est le moyen le plus efficace de suivre plusieurs sources de données, car il fournit un emplacement central pour les entreprises pour surveiller et analyser les performances.\\
  \\
   
Les données sont visualisées sur un tableau de bord sous la forme de tableaux, de graphiques linéaires et de graphiques à barres  afin que les utilisateurs puissent suivre la santé de leur entreprise par rapport aux repères et aux objectifs. \\
Pour la  réalisation de tout ça j'ai utilisé les outils graphiques \textbf{Google} qui sont puissants, simples à utiliser et gratuits.\\
Et j'ai pu intégrer ces outils grâce au bundle \textbf{CMENGoogleChartsBundle}.\\ \\
\begin{center}
\begin{figure}[htp]
  \centering
  \includegraphics[width=12cm]{pg2.png}
  \caption{Tableau de bord.}
  \label{fig:une-autre-image}
\end{figure}
\end{center}


\subsection{Temps réel}

 La surveillance en temps réel réduit les heures d'analyse et la longue ligne de communication qui mettaient auparavant les entreprises au défi.\\  
 La mise en place des tableaux de bord en temps réel est donc  une partie importante du développement du site web.\\ 
 

 La mise en œuvre de ce système n'est pas assez évidente, surtout 
 quand on veut utiliser du \textbf{WebSocket} \\ \\
 
 Malheureusement, la bibliothèque standard \textbf{PHP} ne supporte pas \textbf{WebSocket} , même s'il utilise la communication locale par socket, et il faut encore utiliser un framework.\\
 De même \textbf{Symfony 3 }  n'inclut pas de \textbf{bundle} 
 permettant de faire du temps réel.


\subsubsection{Ajax}
Ce problème de \textbf{WebSocket}, m'a poussé de chercher des
autres solutions.\\ 

J'ai essayé premièrement avec l'\textbf{Ajax} classique, et j'ai 
trouvé pas ma des gens dans la communauté \textbf{Symfony} qui
conseillent d’éviter cette solution.\\
Puisqu'un script qui fait appel à une page php toutes les 5 secondes,il risque que   le serveur d'hébergement perçoit ces demandes comme du spam et bloque le client après un certain temps .\\
Cette solution était facile à implémenter et elle me permet de gagner du temps, mais ses risque m'ont obligés de l'abandonné.\\


\\
J'ai essayé de nouveau avec \textbf{Ajax} mais cette fois ci,avec une autre approche du temps réel, c'est le \textbf{push AJAX}.\\
 Un système \textbf{push AJAX} comprend  un module Apache côté serveur et un bout de script JS côté client. Ce système permet de réaliser des applications temps réel.\\
 J'ai laissé cette solution lors de son test, au motif qu'il n’est pas possible d’envoyer une donnée depuis le serveur directement vers le client.

\subsubsection{WebSocket}
Les difficultés et les problèmes rencontrés avec les solution d'\textbf{Ajax}, m'ont obligé de retourner aux \textbf{Websockets}.\\
\textbf{WebSocket} est une alternative à \textbf{Ajax} plus simple à mettre en œuvre coté client.\\
Contrairement à l'objet \textbf{XMLHttpRequest} d'\textbf{Ajax}, qui envoie des requêtes au serveur et met à jour la page web de façon asynchrone lorsque un script sur le serveur renvoie les résultats, \textbf{WebSocket} permet d'envoyer des données à la page à l'initiative du serveur. On peut donc lancer un traitement sur le serveur qui enverrait des données à un navigateur fonctionnant comme un tableau de bord avec différents \textbf{widgets} qui présenteront les informations reçues.

\\
 \\
Une première implémentation  du \textbf{Websocket} dans le projet, 
est d'utiliser  un adaptateur \textbf{SocketIO} pour PHP, un serveur  \textbf{NodeJS}.\\ 
\textbf{SocketIO},  s’agit d’une bibliothèque JavaScript pour les applications web qui permet le temps réel, elle comporte deux parties, un coté client qui fonctionne dans le navigateur et un coté serveur pour \textbf{Node.js}.\\
Mais cette solution a des conséquences,  quand on arrive aux 
hébergements de l'application, il faudra héberger aussi \textbf{NodeJS}, et ça pourra poser des problèmes avec le serveur d’hébergement qui n'est pas dédié à notre site, et en plus augmentera les charges et dépassera le budget déjà fixer pour l'hébergement.\\ \\


Une deuxième solution , et c'est celle que j'ai choisi, pour intégrer \textbf{Websocket}  dans 
le projet \textbf{Symfony}, est d'utiliser \textbf{Ratchet} .\\
\textbf{Ratchet} est une librairie qui permet de créer des applications temps réelles avec échanges bi-directionnels entre clients et serveur.\\

Elle permet donc de garder une connexion active entre un (ou plusieurs) clients et le serveur. \\
 Cette technologie permet donc d'envoyer les données à tous les clients connectés dès qu'une modification survient, grâce aux \textbf{listeners} du serveur.\\ \\


Actuellement ce système qui permet de réaliser des applications temps réel du site web est en cours de développement.\\

\subsection{Services }

\subsubsection{Alertes et Info}

Pour les alertes et Info, dans \textbf{Symfony} il existe le composant \textbf{SwiftMailer} qui permet de gérer les e-mails. Ce composant contient une classe nommée \textbf{Swift Mailer} qui envoie  les e-mails. \\
 Le conteneur de service de Symfony peut donc accéder à la classe \textbf{Swift Mailer}  grâce au service \textbf{mailer} .\\
 \\
 
 Le concept de service est un bon moyen d'éviter de faire trop de code.\\
 L'avantage des services est que cela permet de bien séparer chaque fonctionnalité de l'application. \\
 Chaque service ne remplit qu'une seule et unique fonction, il est facilement réutilisable, Et on peut  développer, tester et configurer chaque service indépendamment de l'autre. \\ \\
 
Ce service est présent par défaut dans \textbf{Symfony}, donc il est déjà créé, et sa configuration est déjà faite, il ne reste plus qu'à l'utiliser.

\subsubsection{Exportation/Importation}

La bibliothèque \textbf{PHPExcel} est un ensemble de classes pour  PHP, qui  permet d' écrire et de lire à partir de différents formats de feuilles de calcul comme \textbf{Excel}.\\
Il y a deux façons de générer des fichiers Excel dans Symfony 3, soit  \textbf{ExcelBundle} qui est recommandé pour les documents \textbf{Excel}  complexes,   ou  générer des fichiers \textbf{Excel} avec \textbf{Twig} à l'aide de \textbf{TwigExcelBundle}   qui est recommandé pour les documents \textbf{Excel} simples.\\
\\
J'utiliserai donc \textbf{ExcelBundle} pour l'upload et le téléchargement des fichiers \textbf{xlsx} et \textbf{csv}.\\
En ce qui concerne l'impression des données sous forme de fichiers pdf, j'utiliserai le \textbf{bundle}  \textbf{knp-snappy-bundle} qui est basé sur l'outil \textbf{Wkhtmltopdf} qui permet  la génération de PDF.\\

\newpage

\section{Test}

Quand je termine une fonctionnalité je fait des tests pour vérifier qu'elle donne bien le résultat attendu.\\
Les tests sont important et permettent d'assurer le fonctionnement du site web.\\ \\ 
Pour faire un test je :
\begin{enumerate}
 \item  Crée un client HTTP.

 \item   Effectue une requête HTTP sur la page à tester.

\item    S'assure que les éléments sur la page testée sont bien présents .
\end{enumerate}\\
\\

\begin{figure}[htp]
  \centering
  \includegraphics[width=12cm]{t1.png}
  \caption{Page d'accueil.}
  \label{fig:une-autre-image}
\end{figure}

\begin{figure}[htp]
  \centering
  \includegraphics[width=12cm]{t2.png}
  \caption{Section de connexion ou de création de compte.}
  \label{fig:une-autre-image}
\end{figure}

\begin{figure}[htp]
  \centering
  \includegraphics[width=12cm]{t4.png}
  \caption{Page de connexion en responsive design.}
  \label{fig:une-autre-image}
\end{figure}

\begin{figure}[htp]
  \centering
  \includegraphics[width=12cm]{t5.png}
  \caption{Page de création de compte en responsive design.}
  \label{fig:une-autre-image}
\end{figure}

\begin{figure}[htp]
  \centering
  \includegraphics[width=12cm]{t7.png}
  \caption{Page d'administrateur.}
  \label{fig:une-autre-image}
\end{figure}

\begin{figure}[htp]
  \centering
  \includegraphics[width=12cm]{t8.png}
  \caption{Affichages des données admin.}
  \label{fig:une-autre-image}
\end{figure}

\begin{figure}[htp]
  \centering
  \includegraphics[width=12cm]{t10.png}
  \caption{Manipulation des données en responsive design.}
  \label{fig:une-autre-image}
\end{figure}

\begin{figure}[htp]
  \centering
  \includegraphics[width=12cm]{t11.png}
  \caption{Tableau des bords client.}
  \label{fig:une-autre-image}
\end{figure}

 


\newpage
\section{Conclusion}

Ce stage au sein de l'entreprise VMP-CONSULTING termine ses premiers 4 mois , mais en réalité le vrai développement ne fait que commencer.\\

L'analyse des besoins et la rédaction du cahier des charges ont pris énormément de temps   pour être terminé. ils sont la base du projet, quand cette base est solide le projet augmente ses chances de réussite. \\
La rédaction du cahier des charges seule était une expérience importante dans ma vie professionnelle.\\ 
J'ai pris mon temps pour bien le rédiger  afin que tous les besoins de l'entreprise soient détaillés.
  \\ \\

Ce stage est une opportunité pour la réalisation d'un site web de A à Z, il m'a permis d’être le concepteur, le designer, et le développeur d'un projet informatique, mais aussi il est un défi de réaliser toutes ces missions seul. \\ Le manque d'informaticiens à l'entreprise  m'a compliqué la mission, et  m'a obligé des faire plus de recherches et de formations.\\ \\

Ce stage m'a permis d’acquérir des nouvelles connaissances informatiques, et de consolider mes bases. il était une occasion d'explorer le responsive design avec \textbf{Bootstrap}, le JavaScript avec Jquery, et le temps réel avec le WebSocket.\\
En plus Ce site web est  mon premier projet professionnel développé sous \textbf{Symfony}.\\
\\







\newpage
\section{Bibliographie}


\begin{thebibliography}{10}
    
  \beamertemplatebookbibitems
  % Start with overview books.

  \bibitem{www.openclassrooms.com}
   www.openclassrooms.com.
    \newblock {\em cours Symfony}.
    \newblock .
 
     \bibitem{www.symfony.com}
   www.symfony.com.
    \newblock {\em Symfony}.
    \newblock Documentation.
    
      \bibitem{www.symfony.com}
    www.symfony.com.
    \newblock {\em Twig}.
    \newblock Documentation.
    
      \bibitem{www.doctrine-project.org}
   www.doctrine-project.org.
    \newblock {\em Doctrine}.
    \newblock Documentation.

\bibitem{www.symfony.com}
   www.symfony.com.
    \newblock {\em Doctrine}.
    \newblock Documentation.
    
    
    
    
      \bibitem{www.symfony.com}
    www.symfony.com.
    \newblock {\em EasyAdminBundle}.
    \newblock Documentation.
    
      \bibitem{www.symfony.com}
    www.symfony.com.
    \newblock {\em FOSUserBundle}.
    \newblock Documentation.
    
     \bibitem{www.openclassrooms.com}
   www.openclassrooms.com.
    \newblock {\em cour Analyse logiciel avec UML}.
    \newblock .
    
     \bibitem{www.openclassrooms.com}
   www.openclassrooms.com.
    \newblock {\em cour JavaScript}.
    \newblock .
    
     \bibitem{www.openclassrooms.com}
   www.openclassrooms.com.
    \newblock {\em cour JQuery}.
    \newblock .

 \bibitem{www.openclassrooms.com}
   www.openclassrooms.com.
    \newblock {\em cour PHP}.
    \newblock .

 \bibitem{www.openclassrooms.com}
   www.openclassrooms.com.
    \newblock {\em cour Ajax}.
    \newblock .
    
     \bibitem{getbootstrap.com }
    getbootstrap.com .
    \newblock {\em  Boostrap 3}.
    \newblock Documentation.
    
     \bibitem{www.openclassrooms.com}
    www.openclassrooms.com.
    \newblock {\em cours Boostrap}.
    \newblock .
    
     
     \bibitem{http://socketo.me/}
      socketo.me.
    \newblock {\em WebSocket for Php}.
    \newblock .
    
     
     \bibitem{www.developpez.com}
    www.developpez.com.
    \newblock {\em Conception base de données}.
    \newblock UML.
    
    
    
    
    
  \beamertemplatearticlebibitems
  % Followed by interesting articles. Keep the list short. 

  \bibitem{www.everwin.fr}
    www.everwin.fr.
    \newblock Portail web colloboratif.
    \newblock {\em }
   
  \end{thebibliography}


\end{document}
